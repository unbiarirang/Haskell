%! compile with xelatex
\documentclass[11pt, a4paper]{article}

% fonts: Make sure you have SimSun and SimHei installed on your system
\usepackage{xeCJK}
\setCJKmainfont[BoldFont=SimHei]{SimSun}
\setCJKfamilyfont{hei}{SimHei}
\setCJKfamilyfont{kai}{KaiTi}
\newcommand{\hei}{\CJKfamily{hei}}
\newcommand{\kai}{\CJKfamily{kai}}

% style
\usepackage[top=2.54cm, bottom=2.54cm, left=3.18cm, right=3.18cm]{geometry}
\usepackage{indentfirst}
\linespread{1.5}
\parindent 2em
\punctstyle{quanjiao}
\renewcommand{\today}{\number\year 年 \number\month 月 \number\day 日}

\usepackage{amsmath}
\usepackage{amssymb}
\usepackage{xcolor}
\usepackage{hyperref}

\newtheorem{lemma}{引理}
\usepackage[inference]{semantic} % for writing inference rules
\setnamespace{0pt}

\usepackage{enumitem}
\setlist{nolistsep}

\usepackage{empheq}
\newcommand{\boxedeq}[1]{\begin{empheq}[box=\fbox]{align*}#1\end{empheq}}

% A bunch of handy commands
\let\t\texttt
\let\emptyset\varnothing
\let\to\rightarrow
\let\reduce\Rightarrow
\newcommand{\reduceM}{\Rightarrow^{*}}
\let\defas\triangleq
\newcommand{\Bool}{\t{Bool}}
\newcommand{\kword}[1]{{\color{blue} \textsf{#1}}}
\newcommand{\True}{\kword{true}}
\newcommand{\False}{\kword{false}}
\newcommand{\If}{\kword{if}}
\newcommand{\Then}{\kword{then}}
\newcommand{\Else}{\kword{else}}

\title{\textbf{THU Haskell 2019 Homework 08}}
\author{}
\date{\today}

\begin{document}

\maketitle

{\hei 注意:本次作业均为非编程题,请直接在网络学堂提交一份 PDF。前两题为必做题,这两题全部做对即得满分100分。
后两题为选做题,每题25分,视回答情况酌情加分。}

\section{$\lambda_\to$ 扩展}

考虑一个对 $\lambda_\to$ 进行了布尔类型扩展的演算,名为$\lambda_\to^{\Bool}$,其语法定义如下:
\begin{align*}
    \text{Term}~t &::= x & \text{(variable)} \\
    &\mid (t_1~t_2) & \text{($\lambda$-application)} \\
    &\mid (\lambda x:T.t) & \text{($\lambda$-abstraction)} \\
    &\mid \True & \text{(Boolean value true)} \\
    &\mid \False & \text{(Boolean value false)} \\
    &\mid (\If~t_1~\Then~t_2~\Else~t_3) & \text{(if expression)} \\
    \text{(Simple) type}~T &::= \Bool & \text{(Boolean type is the only base type)} \\
    &\mid T_1\to T_2 & \text{(arrow type)}
\end{align*}

注意这里我们新增了三种和布尔相关的项,并人为限制了基本类型(base type)只包含 $\Bool$。优先级和结合性规则同讲义。

新增如下 typing rules:
\begin{align*}
    & \inference[T-True]{}{\Gamma \vdash \True:\Bool} \\
    & \inference[T-False]{}{\Gamma \vdash \False:\Bool} \\
    & \inference[T-If]{\Gamma \vdash t_1:\Bool & \Gamma \vdash t_2:T & \Gamma \vdash t_3:T}{\Gamma \vdash (\If~t_1~\Then~t_2~\Else~t_3):T}
\end{align*}

新增如下 evaluation rules:
\begin{align*}
    &\inference[E-If]{t_1 \reduce t_1'}{(\If~t_1~\Then~t_2~\Else~t_3) \reduce (\If~t_1'~\Then~t_2~\Else~t_3)} \\
    &\inference[E-IfTrue]{}{(\If~\True~\Then~t_2~\Else~t_3) \reduce t_2} \\
    &\inference[E-IfFalse]{}{(\If~\False~\Then~t_2~\Else~t_3) \reduce t_3}
\end{align*}

{\hei 问题:}

1.1\quad 设 $t_0$ 归约到 $\False$ 需要 $k$ 步,不妨记作 $t_0 \reduce_k \False$,那么将 
$$\If~t_0~\Then~((\lambda x:\Bool.x)~t_0)~\Else~((\lambda x:\Bool.\If~x~\Then~\False~\Else~\True)~t_0)$$
归约到 normal form 需要多少步?请写出推导序列(可直接用记号 $t_0 \reduce_k \False$),或者画出证明树来表示归约过程。

1.2\quad 用画证明树的方法(要求标上每一步所使用的规则的名称)证明以下 typing judgment 是成立的:
$$\emptyset, f:\Bool \to \Bool \vdash \lambda x:\Bool.f~(\If~x~\Then~\False~\Else~x) : \Bool \to \Bool$$

1.3\quad 使得 $\Gamma \vdash f~x~y:\Bool$ 成立的 typing context $\Gamma$ 是什么?
请仿照讲义 lambda-calculi-full.pdf 第 61 页对 $\mathbf{S}$ 组合子进行类型推导方法,确定此 $\Gamma$ 最一般化的形式。
{\kai 提示:可以先设 $\Gamma=\emptyset,f:T_f,x:T_x,y:T_y$,其中 $T_f$, $T_x$ 和 $T_y$ 为类型变量,然后找出它们之间的关系,列出类型等式约束,并求出最一般化的解。}

\section{类型推断}

讲义 lambda-calculi-full.pdf 第 61 -- 70 页介绍的 Hindley-Milner 类型推导算法适用于 System F。
而 Haskell 语言的基础就是 System F。因此,我们可以运用该类型推导算法的思想,对 Haskell 的表达式进行类型推导。
请手工推导出以下 Haskell 表达式最具一般性的类型,要求简要说明推理过程,即,至少要说明你列了哪些有关类型约束的等式以及每个等式的依据(由于我们在 Haskell 中,请使用 \t{e :: t} 形式的 typing judgment):

2.1\quad \t{(\$).(\$)}

{\kai 提示:\t{(\$) (\$)} 的类型与 \t{(\$).(\$)} 一致,尝试推断 \t{(\$) (\$)} 的类型可以启发你完成本题。求出类型后想一想:把其中的类型变量替换为原子命题,把 \t{->} 替换为逻辑蕴含,这个命题公式是有效式吗?}

2.2\quad \t{(.)\$(.)}

{\kai 提示:\t{(.)\$(.)} 的类型与 \t{(.) (.)} 一致,尝试推断 \t{(.) (.)} 的类型可以启发你完成本题。求出类型后想一想:把其中的类型变量替换为原子命题,把 \t{->} 替换为逻辑蕴含,这个命题公式是有效式吗?}

2.3\quad \t{\textbackslash c f -> if c then map f else tail}

\section{扩展系统的元性质判定(选做题)}

{\kai 提示:本题工作量小,但是有一定的思维难度。根据给出的判定结果是否正确来评分。}

不难证明,$\lambda_\to^{\Bool}$ 和 $\lambda_\to$ 一样,也满足如下三个元性质:

\begin{lemma}
    (Determinism of reduction) If $t \reduce t_1$ and $t \reduce t_2$, then $t_1 = t_2$.
\end{lemma}

\begin{lemma}
    (Progress) For any term $t$ and type $T$, if $\vdash t : T$,
    then either $t$ is a value, or $t \reduce t'$ for some term $t'$.
\end{lemma}

\begin{lemma}
    (Preservation) If $\vdash t:T$ and $t \reduce t'$, then $\vdash t':T$.
\end{lemma}

但是,在 $\lambda_\to^{\Bool}$ 的基础上做一些扩展,那么这三个元性质就不一定成立了。请分别考虑以下几种不同的扩展方法(都是在 $\lambda_\to^{\Bool}$ 上扩展,这几种扩展之间是正交的),判断三个元性质中,哪些变得不成立了。对于不成立的性质,请各举出一个具体的反例。对于仍然成立的性质,你无需给出证明。

3.1\quad 新增项 \t{foo},以及 evaluation rules
\begin{align*}
    & \inference[E-Foo1]{}{(\lambda x:\Bool.x) \reduce \t{foo}} \\
    & \inference[E-Foo2]{}{\t{foo} \reduce \True}
\end{align*}

3.2\quad 去掉 evaluation rule E-App1

3.3\quad 新增 typing rule
$$
    \inference[T-FunnyApp]{\Gamma\vdash t_1:\Bool\to\Bool\to\Bool & \Gamma\vdash t_2:\Bool}
        {\Gamma\vdash (t_1~t_2):\Bool}
$$

3.4\quad 新增 typing rule
$$
    \inference[T-FunnyAbs]{}{\vdash (\lambda x:\Bool.t) : \Bool}
$$

3.5\quad 新增 typing rule
$$
    \inference[T-FunnyApp']{\Gamma\vdash t_1:\Bool & \Gamma\vdash t_2:\Bool}{\Gamma\vdash (t_1~t_2):\Bool}
$$

\section{System F 的元性质证明(选做题)}

{\kai 提示:本题工作量较大,但只要按部就班地开展证明也不难。根据证明的完整性评分。}

与 $\lambda_\to$ 类似,$F$ 也具有 progress 和 preservation 的元性质:

\begin{lemma}
    ($F$-progress) For any term $t$ and type $T$, if $\vdash t : T$,
    then either $t$ is a value, or $t \reduce t'$ for some term $t'$.
\end{lemma}

\begin{lemma}
    ($F$-preservation) If $\vdash t:T$ and $t \reduce t'$, then $\vdash t':T$.
\end{lemma}

请先阅读 \textit{Types and Programming Languages} 定理 9.3.5 ($\lambda_\to$-progress) 以及引理 9.3.4 的证明,还有定理 9.3.9 ($\lambda_\to$-preservation) 的证明,仿照该方法证明 $F$ 满足上述两个元性质。尽你所能给出所有必要的证明细节。注:你可以采用不同于讲义中的符号系统,但请在证明之前给出所有要用到的定义和规则。

\end{document}
